\documentclass{article}
%\usepackage{verbatim}
\begin{document}

\title{Analysis}
\author{J. L. De Coi}
\maketitle

\begin{itemize}
	\item The following Java-like code tries to catch the logic of the algorithm: the synchronous protocol, as well as other implementation issues entailed by the choice of Java as a language, is not supposed to be necessarily part of the final solution
	\item Comments are written over the line(s) they refer to
	\item
\begin{verbatim}
Type1[] t1a;
Type2[] t2a=t1a.method();

\end{verbatim}
is a shortcut for
\begin{verbatim}
Type1[] t1a;
Type2[] t2a= new Type2[t1a.length];
for(i=0; i<t1a.length; i++) t2a[i]=t1a[i].method();
\end{verbatim}
\end{itemize}

\begin{verbatim}

class NegotiationEngine{

  Policy p;
  Status s;
  /* Checks whether the actions, which the Peer claims to have
  performed
  * were actually performed and
  * were performed in the right way */
  Checker c;
  TerminationAlgorithm ta;
  NegotiationStrategy ns;
  /* Let think at this abstraction level that the two Peers negotiate
  directly. Afterwards suitable wrappers will be developed */
  NegotiationEngine other;

  void perform(FilteredPolicy receivedFp, Notification[] receivedNa){
      s.add(receivedFp);
      s.add(receivedNa);
    Action[] aa=receivedNa.getActions();
    /* In order to report whether an action was performed or not, a
    boolean could be enough. The class PerformedCheck was defined in
    order to allow more information exchange. E.g. a PerformedCheck
    object could contain not only the notification whether the action
    was actually executed or not, but also the notification whether
    the check itself was performed or the other Peer's notification
    was simply trusted */
    PerformedCheck[] pca=c.checkPerformed(aa);
      s.add(pca);
    // Only for actions which were actually performed
    WellPerformedCheck[] wpca=c.checkWellPerformed(aa);
      s.add(wpca);
    if(!ta.terminate(s)){
      Action[] toBePerformed=ns.identifyActions(s, p);
      toBePerformed.perform();
      Notification[] toSendNa=new Notification[](toBePerformed);
        s.add(toSendNa);
      FilteredPolicy toSendFp=filterPolicy();
        s.add(toSendFp);
      other.perform(toSendFp, toSendNa);
    }
  }
  
  /* As far as I have understood, there is just one way to filter
  policies, therefore there is no need to define a new Class taking
  care of the filtering process, since it can be carried out inside
  the class NegotiationEngine */
  FilteredPolicy filterPolicy(){
    //...
  }

}

\end{verbatim}

\end{document}