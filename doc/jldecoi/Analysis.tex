\documentclass{article}
\usepackage{verbatim}
\begin{document}

\title{Analysis}
\author{J. L. De Coi}
\maketitle

\begin{itemize}
	\item The following Java-like code tries to catch the logic of the algorithm: the synchronous protocol, as well as other implementation issues entailed by the choice of Java as a language, is not supposed to be necessarily part of the final solution
	\item Comments are written over the line(s) they refer to
	\item
\begin{verbatim}
Type1[] t1a;
Type2[] t2a=t1a.method();

\end{verbatim}
is a shortcut for
\begin{verbatim}
Type1[] t1a;
Type2[] t2a= new Type2[t1a.length];
for(i=0; i<t1a.length; i++) t2a[i]=t1a[i].method();
\end{verbatim}
\end{itemize}

\section{Peer}

\begin{verbatim}

Status s; // Initialised to an empty status
Checker c; // It is not included in the negotiation model
TerminationAlgorithm ta;
Filter f;
CredentialSelectionFunction csf;
Policy p;
Peer otherPeer;

void perform(FilteredPolicy fp, Notification[] na){
  Action[] unlocked;
  FilteredPolicy[] fpToSend;
  
  s.add(fp);
  s.add(na);
  Check[] ca = c.check(na);
  
  // It is not explicitely included in the negotiation model
  s.add(ca);
  
  if(ta.terminate(s)){
    send(TERMINATION_MESSAGE, otherPeer);
    return;
  }
  
  /* Methods extractActions and isLocked are shared among
  each Filter subclass. Indeed they depend only on the
  coding of the Actions, i.e. on the Protune language */
  Action[] toPerform = f.extractActions(fp);
  
  /* Why do I filter policies even for unlocked actions? */
  FilteredPolicy[] myFp = f.filter(toPerform, p, s);
  
  for each i
    /* Why should I write f.isLocked(myFp[i]) instead of
    f.isLocked(toPerform[i])? */
    if(f.isLocked(myFp[i])) fpToSend.add(myFp[i]);
    else unlocked.add(toPerform[i]);

  Action[] aa = csf.selectActions(unlocked, s);
  Notification[] naToSend = aa.perform();
  s.add(fpToSend);
  s.add(naToSend);
  
  Message m = new Message(fpToSend, naToSend);
  send(m, otherPeer);
}

\end{verbatim}

\section*{Main changes}

\begin{itemize}
\item Entities \texttt{Checker} and \texttt{Policy} were added
\item \texttt{evaluationState} was removed: it is part of the \texttt{InferenceEngine} and not of the \texttt{Peer}
\item Method \texttt{terminate} of class \texttt{TerminationAlgorithm} has exactly one parameter of type \texttt{Status}
\item Method \texttt{filter} of class \texttt{Filter} has exactly three parameters of type \texttt{Action}, \texttt{Policy} and \texttt{Status}
\item Method \texttt{selectActions} of class \texttt{CredentialSelectionFunction} has exactly two parameters of type \texttt{Action[]} and \texttt{Status}
\end{itemize}

\section{Filter}

\begin{verbatim}

InferenceEngine ie;

filter(Action a, Policy p, Status s){
  Action[] aa = ie.firstStep(a, p, s);
  while(aa.length != 0){
    Notification[] na = aa.perform();
    ie.addToEvaluationState(na);
    aa = ie.secondStep(a, p, s);
  }
  return ie.thirdStep(a, p, s);
  // What about the actions whose evaluation is delayed?
}

\end{verbatim}

\end{document}