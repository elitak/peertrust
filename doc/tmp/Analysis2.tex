\documentclass{article}
\usepackage{verbatim}
\begin{document}

\title{Analysis}
\author{J. L. De Coi}
\maketitle

\begin{itemize}
	\item The following Java-like code tries to catch the logic of the algorithm: the synchronous protocol, as well as other implementation issues entailed by the choice of Java as a language, is not supposed to be necessarily part of the final solution
	\item Comments are written over the line(s) they refer to
	\item
\begin{verbatim}
Type1[] t1a;
Type2[] t2a=t1a.method();

\end{verbatim}
is a shortcut for
\begin{verbatim}
Type1[] t1a;
Type2[] t2a= new Type2[t1a.length];
for(i=0; i<t1a.length; i++) t2a[i]=t1a[i].method();
\end{verbatim}
\end{itemize}

\begin{verbatim}

Status s;
NegotiationStrategy ns;
TerminationAlgorithm ta;
Filter f;
Peer otherPeer;

void perform(FilteredPolicy fp, Notification[] na){
  Check[] c = na.check();
  // If c[i] failed, it is not put in the state
  s.add(c);
  evaluationState.add(c);
  if(ta.terminate()){
    send(TERMINATION_MESSAGE, otherPeer);
    return;
  }
  Actions[] toPerform = f.extractActions(fp);
  Actions[] unlocked;
  FilteredPolicy[] fpToSend;
  FilteredPolicy[] myFp = f.filter(toPerform);
  for each i
    if(f.isLocked(myFp[i])) fpToSend.add(myFp[i]);
    else add the corresponding action to unlocked;
  Action[] aa = ns.selectActions(unlocked);
  Notification[] naToSend = aa.perform();
  Message m = new Message(fpToSend, naToSend);
  send(m, otherPeer);
}

\end{verbatim}

\begin{comment}

filtering(Action a){
   ProtuneNegEngine pne = new ProtuneNegEngine();
   myAa = pne.f1(a);
   while(myAa is not emptySet){
     A = execute(myAa);
     myAa = pne.f2(A);
   }
   return pne.f3(A);
}

\end{comment}

\end{document}