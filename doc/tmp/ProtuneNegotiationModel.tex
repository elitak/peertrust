\documentclass{article}
\usepackage{verbatim}
\begin{document}

\title{Protune Negotiation Model}
\author{J. L. De Coi, D. Ghita, D. Olmedilla}
\maketitle

\begin{comment}
\begin{description}
\item[Definition (List)] A list of elements $l$ is
  \begin{itemize}
	\item either the empty list
	\item or an element followed by a list
  \end{itemize}
\end{description}

Let $l$ be a non-empty list, therefore it is composed of an element $e$ and a (sub)list $l'$. $e$ (resp. $l'$) is called \textit{head} (resp. \textit{tail}) of $l$.

We will write the empty list as $[]$ and a non-empty list as $[head, tail]$.

\begin{description}
\item[Definition (Length of a list)] The length of a list is a function
  \begin{displaymath}
  length: L \rightarrow N
  \end{displaymath}
  where
  \begin{itemize}
	\item $L$ is the set of all lists
	\item $N$ is the set of all natural numbers
  \end{itemize}
  such that
  \begin{itemize}
	\item $length([])=0$
	\item $length([head, tail])=1+length(tail)$
  \end{itemize}
\item[Definition (n-th element of a list)] Let define the function
  \begin{displaymath}
  elementAt: A \subset (L \times (N \backslash \{0\})) \rightarrow E
  \end{displaymath}
  where
  \begin{itemize}
	\item $N$ is the set of all natural numbers
	\item $L$ is the set of all lists
	\item $E$ is the set of all elements
  \end{itemize}
	such that
  \begin{itemize}
	\item $elementAt(1, [head, tail])=head$
	\item $elementAt(n, [head, tail])=elementAt(n-1, tail)$
  \end{itemize}
  Notice that such function is not defined for ordered pairs $(n, l)$ such that $n > length(l)$.
\item[Definition  (Negotiation Message)] A negotiation message $m$ is an ordered pair
  \begin{displaymath}
  (fp, C)
  \end{displaymath}
  where
  \begin{itemize}
	\item $fp \equiv$ a filtered policy
	\item $C \equiv$ a set of credentials
  \end{itemize}
\item[Definition  (Negotation State)] A negotiation state $s$ is an ordered pair
  \begin{displaymath}
  (M_{snd}, M_{rcv})
  \end{displaymath}
  where
  \begin{itemize}
	\item $M_{snd} \equiv$ list of sent messages
	\item $M_{rcv} \equiv$ list of received messages
  \end{itemize}
\item[Definition (Transition system)] A transition system $TS$ is a tuple
  \begin{displaymath}
  (S, M_{snd}, M_{rcv},  s_{0}, f_{snd}, f_{rcv})
  \end{displaymath}
  where
  \begin{itemize}
	\item $S \equiv$ a set of states
	\item $M \equiv$ a set of messages
	\item $s_{0} \equiv$ a state (calles \textit{initial state})
	\item $f_{snd} \equiv$ a function $S \times M_{snd} \rightarrow S$
	\item $f_{rcv} \equiv$ a function $S \times M_{rcv} \rightarrow S$
  \end{itemize}
  Where $f_{snd}$ and $f_{snd}$ are callsd \textit{negotiation functions}.
\end{description}

In the following we will consider only these particular negotiation functions

\begin{displaymath}
f_{snd}:(s_{snd}, m) \rightarrow s'_{snd}
\end{displaymath}

where

\begin{itemize}
	\item $s_{snd} = (M_{snd}, M_{rcv})$
	\item $s'_{snd} = ([m, M_{snd}], M_{rcv})$
\end{itemize}

and

\begin{displaymath}
f_{rcv}:(s_{rcv}, m) \rightarrow s'_{rcv}
\end{displaymath}

where

\begin{itemize}
	\item $s_{rcv} = (M_{snd}, M_{rcv})$
	\item $s'_{rcv} = (M_{snd}, [m, M_{rcv}])$
\end{itemize}

\begin{description}
\item[Definition (Credentials selection)] A credentials selection $CS$ is a function
  \begin{displaymath}
  P \times S \rightarrow C
  \end{displaymath}
  where
  \begin{itemize}
	\item $P$ is the set of all policies
	\item $S$ is the set of all states
	\item $C$ is the set of all credentials
  \end{itemize}
\item[Definition (Filter)] A filter $F$ is a function
  \begin{displaymath}
  P \times S \rightarrow P
  \end{displaymath}
  where
  \begin{itemize}
	\item $P$ is the set of all policies
	\item $S$ is the set of all states
  \end{itemize}
\item[Definition (Negotiation Strategy)] A negotiation strategy $NS$ is an ordered pair
  \begin{displaymath}
  (cs, f)
  \end{displaymath}
  where
  \begin{itemize}
	\item $cs$ is a credentials selection function
	\item $f$ is a filter
  \end{itemize}
\item[Definition (Termination Algorithm)] A termination algorithm $TA$ is a function
  \begin{displaymath}
  S \rightarrow \{true, false\}
  \end{displaymath}
\item[Definition (Negotiation Model)] A negotation model $NM$ is a tuple
  \begin{displaymath}
  (p, C, TS, ns, ta)
  \end{displaymath}
  where
  \begin{itemize}
	\item $p$ is a policy
	\item $C$ is a set of credentials
	\item $TS$ is a transition system
	\item $ns$ is a negotiation strategy
	\item $ta$ is a termination algorithm
  \end{itemize}
\end{description}
\end{comment}

\begin{description}
\item[Definition 1 (List)] A list of elements is
  \begin{itemize}
	\item either the empty list
	\item or an element followed by a list
  \end{itemize}
\end{description}

Let $l$ be a non-empty list, therefore it is composed of an element $e$ and a (sub)list $l'$. $e$ (resp. $l'$) is called \textit{head} (resp. \textit{tail}) of $l$.

We will write the empty list as $[]$ and a non-empty list as $[head, tail]$.

\begin{description}
\item[Definition 2 (Length of a list)] The length of a list is a function
  \begin{displaymath}
  length: L \rightarrow {\cal N}
  \end{displaymath}
  where
  \begin{itemize}
	\item $L$ is the set of all lists
	\item ${\cal N}$ is the set of all natural numbers
  \end{itemize}
  such that
  \begin{itemize}
	\item $length([])=0$
	\item $length([head, tail])=1+length(tail)$
  \end{itemize}
\item[Definition 3 ($i$-th element of a list)] The $i$-th element of a list is a function
  \begin{displaymath}
  elementAt: A \rightarrow E
  \end{displaymath}
  where
  \begin{itemize}
	\item $A = \{ (n, l) \in {\cal N}^{*} \times L : n \leq length(l) \}$
	\item ${\cal N}^{*}$ is ${\cal N}\ \backslash \ \{0\}$
	\item $E$ is the set of all elements
  \end{itemize}
	such that
  \begin{itemize}
	\item $elementAt(1, [head, tail])=head$
	\item $elementAt(i, [head, tail])=elementAt(i-1, tail)$
  \end{itemize}
\item[Definition 4 (Containment relationship)] Let $e$ be an element and $l$ be a list. $l$ \textit{contains} $e$ iff $e=elementAt(i, l)$ for some $i$.
\end{description}

If $e$ is contained in $l$ we will write (with abuse of notation) $e \in l$.

\begin{description}
\item[Definition 5 (Negotiation Message)] A negotiation message is an ordered pair
  \begin{displaymath}
  (fp, C)
  \end{displaymath}
  where
  \begin{itemize}
	\item $fp \equiv$ a filtered policy
	\item $C \equiv$ a set of credentials
  \end{itemize}
\item[Definition 6 (Negotation State)] A negotiation state is an ordered pair
  \begin{displaymath}
  (M_{snd}, M_{rcv})
  \end{displaymath}
  where both $M_{snd}$ and $M_{rcv}$ are lists of messages.
\end{description}

$M_{snd}$ (resp. $M_{rcv}$) is intended to represent the list of sent (resp. received) messages.

Notice that for each state $s=(M_{snd}, M_{rcv})$ the following expressions represent the sets of credentials globally sent or received so far

\begin{displaymath}
\bigcup \{C_{i} : \exists fp_{i}\ (fp_{i}, C_{i}) \in M_{snd}\}
\end{displaymath}

\begin{displaymath}
\bigcup \{C_{i} : \exists fp_{i}\ (fp_{i}, C_{i}) \in M_{rcv}\}
\end{displaymath}

\begin{description}
\item[Definition 7 (Peer)] A Peer is composed of the following elements
  \begin{itemize}
	\item A policy $p$
	\item A set of credentials $C$
	\item A set of filtered policies $FP$
	\item A set of states $S$
	\item A state $s_{0}$ (called \textit{initial state})
	\item Two sets of messages $M_{snd}$ and $M_{rcv}$
	\item A function $tf_{snd}:S \times M_{snd} \rightarrow S$ (called \textit{transition function})
	\item A function $tf_{rcv}:S \times M_{rcv} \rightarrow S$ (called \textit{transition function})
	\item A function $f:S \rightarrow FP$ (called \textit{filter})
	\item A function $csf:S \rightarrow {\cal P}(C)$ (called \textit{credential selection function})
	\item A function $ta:S \rightarrow \{true, false\}$ (called \textit{termination algorithm})
  \end{itemize}
\end{description}

The tuple $(S, s_{0}, M_{snd}, M_{rcv}, tf_{snd}, tf_{rcv})$ is also called \textit{transition system}, the ordered pair $(csf, ta)$ is also called \textit{negotiation strategy}.

The intended meaning is as follows

\begin{itemize}
	\item $p$ represents the Peer's policy protecting the local credentials and allowing access to the local resources
	\item $C$ represents the set of the credentials local to the Peer
	\item $FP$ represents the set of filtered policies the Peer can send to the other Peer
	\item $S$ represents the set of states in which the Peer can be
	\item $s_{0}$ represents the initial state, i.e. the state in which the Peer is at the beginning of the negotiation
	\item $M_{snd}$ (resp. $M_{rcv}$) represents the set of messages the Peer can send (resp. receive)
	\item $tf_{snd}$ (resp. $tf_{rcv}$) represents the Peer's state transition following the sending (resp. reception) of a message
	\item $f$ represents the process of filtering the Peer's policy according to the current state
	\item $cfs$ represents the process of selecting the Peer's credentials to send to the other Peer
	\item $ta$ represent the Peer's decision about whether going on or terminating the current negotiation
\end{itemize}

Hereafter we will consider Peers with the following characteristics

\begin{itemize}
\item $s_{0}$ is empty (i.e. it is the ordered pair $([], [])$)
\item $f_{snd}:(s_{snd}, m) \rightarrow s'_{snd}$ where
  \begin{itemize}
	\item $s_{snd} = (M_{snd}, M_{rcv})$
	\item $s'_{snd} = ([m, M_{snd}], M_{rcv})$
  \end{itemize}
\item $f_{rcv}:(s_{rcv}, m) \rightarrow s'_{rcv}$ where
  \begin{itemize}
	\item $s_{rcv} = (M_{snd}, M_{rcv})$
	\item $s'_{rcv} = (M_{snd}, [m, M_{rcv}])$
  \end{itemize}
\end{itemize}

\begin{description}
\item[Definition 8 (Negotiation)] A Negotiation is an ordered pair
  \begin{displaymath}
  (P^{1}, P^{2})
  \end{displaymath}
  where $P^{1}$ (resp. $P^{2}$) is a Peer.
\end{description}

$P^{1}$ (resp. $P^{2}$) is called \textit{requester Peer} (resp. \textit{provider Peer}).

Hereafter we will consider Negotiations with the following characteristics

\begin{itemize}
\item $M_{snd}^{1} = M_{rcv}^{2}$
\item $M_{snd}^{2} = M_{rcv}^{1}$
\end{itemize}

\end{document}